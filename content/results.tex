\part{Implementation and Numerical Results}






\chapter{Computer Implementation}

\section{Numerical Integration of Sommerfeld Integrals}

\section{Discrete Complex Image Method}

\subsection{Fitting Algorithm}

\cite{Hua1989}
\cite{Sarkar1995}
\cite{mohammadi-ghazi2016}
\cite{vandenberg2008, spgl1site}

\section{Fast Multipole Method}

\subsection{Geometrical Definition}

The following considerations concerning the \ac{FMM} use the geometrical
assumptions elucidated by \cref{fig:fmm_geometry}.



Within the scope of the present thesis the grouping scheme is chosen
according to \cref{fig:fmm_geometry},~\ie, all groups are cubes with side
length $w$.
Towards an inclusion of a ground half-space at $z < 0$ the whole grouping
configuration is aligned with $z = 0$ as shown.



This notation is in accordance to the popular prescription of the \ac{FMM}
algorithm by \textcite{Coifman1993}.

\begin{figure}
	\centering
	\begin{tikzpicture}
		[scale = 0.7]
		
		\newcommand{\w}{3}
		\newcommand{\dotsize}{1.5pt}
		
		\coordinate(r_s) at (1.5*\w,0.5*\w);
		\coordinate(r_o) at ($(r_s) + (2*\w,\w)$);
		\coordinate(r_prime) at ($(r_s) + (0.2*\w,0.45*\w)$);
		\coordinate(r) at ($(r_o) + (-0.3*\w,0.3*\w)$);
		
		% Ground
		\fill[color = blue!30!white] (-0.5*\w,0) rectangle (4.5*\w,-1.5*\w);
		
		% Coordinate system
		\draw [color=gray!20]  [step=\w] (-0.5*\w,-1.5*\w) grid (4.5*\w,2.5*\w);
		\draw[->,>=stealth] (-0.75*\w,0) -- (4.75*\w,0) node[right] {$\left(x,y\right)$};
		\draw[->,>=stealth] (0,-1.75*\w) -- (0,2.75*\w) node[above] {$\left(z\right)$};
		\foreach \c in {1,2,3,4}{\draw (\c*\w,-.1) -- (\c*\w,.1) node[below=4pt] {$\c w$};}
		\foreach \c in {1,2}{\draw (-0.1,\c*\w) -- (0.1,\c*\w) node[left=4pt] {$\c w$};}
		\draw (-0.1,-0.1) node[below left] {$0$};
		
		% Vectors
		\draw[->, >=stealth] (r_s) -- node[below right]{$\fmmDiffGroupCenters$} (r_o);
		\draw[->, >=stealth, line width=1.0pt, red] (r_prime) -- node[above left]{$\rdiff$} (r);
		\draw[->, >=stealth] (r_prime) -- (r_s);
		\draw[->, >=stealth] (r_o) -- (r);
		
		% Points
		\fill (r) circle (\dotsize) node[above]{$\robs$};
		\fill (r_prime) circle (\dotsize) node[above]{$\rsrc$};
		\fill (r_s) circle (\dotsize) node[below]{$\fmmSrcGroupCenter$};
		\fill (r_o) circle (\dotsize) node[below right]{$\fmmObsGroupCenter$};
		
		% Group boxes
		\draw($(r_s) - (\w/2,\w/2)$) rectangle ($(r_s) + (\w/2,\w/2)$);
		\draw($(r_o) - (\w/2,\w/2)$) rectangle ($(r_o) + (\w/2,\w/2)$);
		
	\end{tikzpicture}

	\caption[\acs{FMM} geometry]
	{Geometric definition of the \ac{FMM} implementation.}
	\label{fig:fmm_geometry}
\end{figure}

Within this thesis the goal is not to develop a fully-featured \ac{FMM}
implementation, \eg, for the use in an electromagnetic solver.
It is rather desired to have a simple but flexible test code to examine
different possibilities to adapt the \ac{FMM} to the half-space Green's
function.
With this key requirement in mind an experimental version of the \ac{FMM}
is implemented which is restricted in the following aspects.
\begin{itemize}
	\item Only the scalar case is considered.
	\item Only $\diracDelta$-functions are used as basis and testing functions.
	\item The locations of basis and testing functions are set arbitrarily or 
	according to a specific test case. No real body-like structures are
	considered.
	\item Near interactions are not considered as their conventional \ac{MoM}
	treatment is straightforward.
\end{itemize}

Even though the described setup is conceptually similar to the examples
provided in a paper by \textcite{Hansen2013}, it is found that the subject
of this thesis requires a more versatile approach.
Therefore, the above specifications are merged into the prescription of the
algorithm given by the popular paper by \textcite{Coifman1993}.

Throughout this thesis the regular grouping scheme elucidated by
\cref{fig:fmm_geometry} is assumed for all considerations regarding the \ac{FMM}.
All groups are cubical with side length $w$ and uniquely identified by a
multiindex $\fmmMultiIndex \in \Z^3$ defined as
\begin{equation}
	\fmmMultiIndex \left(\pvec{r}\right) \coloneqq \lfloor \pvec{r} / w \rfloor
\end{equation}
where the floor function is defined to operate component-wise on the vector
$\pvec{r}$ given in Cartesian coordinates.

The centers of certain source and observation groups are denoted by 
$\fmmGroupCenter{\fmmMultiIndex^\prime}$ and 
$\fmmGroupCenter{\fmmMultiIndex}$, respectively.
The connection between two group centers is denoted by
$\fmmDiffGroupCenters{\fmmMultiIndex}{\fmmMultiIndex^\prime} \coloneqq \fmmGroupCenter{\fmmMultiIndex} - \fmmGroupCenter{\fmmMultiIndex^\prime}$.

Within each group the source or observation points belonging to this group
are consecutively numbered by an index $\alpha \in \N$.
Globally, all points are identified by a unique index $n \in \N$.
The small example given in \cref{tab:fmm_grouping} illustrates
the chosen indexing scheme.

The implemented version of the \ac{FMM} does impose no restrictions to the
locations of basis and testing functions other than the fulfillment of the
separation criterion which makes the \ac{FMM} applicable.
The basis and testing functions are especially not assumed to be arranged
according to Galerkin's method \cite[p.~7]{Harrington1993}, where the
self-coupling case regularly occurs.

Due to this freedom it turns out to be advantageous to establish two separate
correspondences as the one given by \cref{tab:fmm_grouping}: one for source
points and one for observation points.

\begin{table}[hbt]
	\centering
	\begin{tabular}{ccc}
		\toprule%
		$\fmmMultiIndex$ & $\alpha$ & $n$ \\
		\midrule
		$\left(0, 0, 0\right)$ & $1$ & $1$ \\
		                       & $2$ & $2$ \\
		                       & $3$ & $3$ \\
		\midrule
		$\left(5, 0, 2\right)$ & $1$ & $4$ \\
		                       & $2$ & $5$ \\
		                       & $3$ & $6$ \\
		                       & $4$ & $7$ \\
		\midrule
		$\left(0, 3, 0\right)$ & $1$ & $8$ \\
		                       & $2$ & $9$ \\
		                       & $3$ & $10$ \\
		\bottomrule
	\end{tabular}

	\caption[\acs{FMM} grouping scheme]
	{\acs{FMM} grouping scheme. The setup in this example has three groups
	containing three, four, and another three points, respectively.}

	\label{tab:fmm_grouping}

\end{table}








\chapter{Numerical Results}
\label{ch:numerical_results}

All results are computed using \texttt{double} precision floating point numbers.

Throughout this chapter relative errors are computed according to
\begin{definition}[Relative error]
    Let $x_\mathrm{r} \neq 0$ be a reference value.
    Then the \emph{relative error} of a result $x_\mathrm{n}$ with respect to
    $x_\mathrm{r}$ is defined by 
    \begin{equation}
        \epsilon \left( x_\mathrm{n} \right) \coloneqq
        20 \log_{10} \abs{\frac{x_\mathrm{n} - x_\mathrm{r}}{x_\mathrm{r}}}
        \SI{}{\decibel}\, .
    \end{equation}
\end{definition}


\begin{table}[hbt]
	\centering
	\begin{tabular}{ccc}
		\toprule%
		Name & $\epsr^\prime$ & $\sigma$ \\
		\midrule
		Dry ground & $\num{3}$  & $\SI{0.1}{\milli\siemens\per\metre}$ \\
		Seawater   & $\num{81}$ & $\SI{4.0}{\siemens\per\metre}$ \\
		\bottomrule
	\end{tabular}

	\caption[Example half-space environments]
	{Example half-space environments used for numerical examples.
    Both environments are considered at $f = \SI{10}{\mega\hertz}$.}
	\label{tab:numerical_examples_half_spaces}
\end{table}


\section{Numerical Integration of Sommerfeld Integrals}

\cite{Michalski2016a}

Asymptotic behavior \cite{Michalski1998}
\begin{equation}
    \tilde{g}\left(\xi\right)
    \propto \frac{\exp{-\zeta \xi}}{\xi^\alpha}
    \left(C + \mathcal{O}\left(\xi^{-1}\right)\right)
\end{equation}

For Sommerfeld identity: $\xi = \abs{z}$, $\alpha = 1/2$

\begin{figure}
    \centering
    \begin{subfigure}[]{0.47\textwidth}
        \begin{tikzpicture}
            \pgfplotsset{small}
            \begin{axis}[
                width=\textwidth,% subfigure textwidth
                colormap/jet,
                colorbar horizontal,
                colorbar style = {xlabel = {Relative error in $\SI{}{\decibel}$}},
                unbounded coords = jump,
                xlabel = {$\rho / \lambda_0$},
                ylabel = {$z / \lambda_0$},
                xmode = log,
                ymode = log,
                view = {0}{90},
                ]
                \addplot3 [
                    surf,
                    mesh/ordering=y varies,
                    ] table[
                        x = x_by_lambda_0, y = z_by_lambda_0, z = rel_err_db,
                    ]
                    {thesis_somm_id_ref_time.dat};
            \end{axis}
        \end{tikzpicture}
        \label{fig:somm_id_ref_time:sub1}
        \caption{Relative error. The maximum error is $\SI{-134.42}{\decibel}$.
        At the missing points numerical integration is exact down to machine
        precision.}
    \end{subfigure}
    \begin{subfigure}[]{0.47\textwidth}
        \begin{tikzpicture}
            \pgfplotsset{small}
            \begin{axis}[
                width=\textwidth,% subfigure textwidth
                colormap/jet,
                unbounded coords = jump,
                colorbar horizontal,
                colorbar style = {
                    xlabel = {Relative execution time},
                    xticklabel = {$10^{\pgfmathprintnumber{\tick}}$},
                    },
                xlabel = {$\rho / \lambda_0$},
                ylabel = {$z / \lambda_0$},
                xmode = log,
                ymode = log,
                view = {0}{90},
                ]
                \addplot3 [
                    surf,
                    mesh/ordering=y varies,
                    ] table [
                        x = x_by_lambda_0,
                        y = z_by_lambda_0,
                        z expr=log10(\thisrow{rel_time}),
                        ]
                        {thesis_somm_id_ref_time.dat};
            \end{axis}
        \end{tikzpicture}
        \label{fig:somm_id_ref_time:sub2}
        \caption{Execution time relative to the point with the fastest
        execution.
        A total of $\num{50}$ evaluations of the Sommerfeld integral are timed
        at each point $\left(\rho,z\right)$.}
    \end{subfigure}

    \caption[Numerical evaluation of the Sommerfeld identity integral]
    {Numerical evaluation of the Sommerfeld identity with the source at
    the origin for $10^{-1} \leq a / \lambda_0 \leq 10^{3}$ where
    $a \in \left\{ \rho, z \right\}$.}
    \label{fig:somm_id_ref_time}
\end{figure}







\begin{figure}
    \newcommand{\sizeFactor}{0.47}
    \centering
    \begin{subfigure}[t]{\textwidth}
        \begin{tikzpicture}
            \pgfplotsset{small}
            \matrix {
            \begin{axis}[
                width=\sizeFactor\textwidth,
                height=\sizeFactor\textwidth,
                colormap/jet,
                colorbar horizontal,
                unbounded coords = jump,
                colorbar style = {xlabel = {$20\log_{10}\abs{G_\mathrm{tot}}$}},
                xlabel = {$\rho / \lambda_0$},
                ylabel = {$z / \lambda_0$},
                view = {0}{90},
            ]
            \addplot3 [
                surf,
                point meta max=-26, point meta min=-97,
                mesh/ordering=y varies,
                ] table[
                    x = x_by_lambda_0, y = z_by_lambda_0, z = g_tm_full_log,
                ]
                {thesis_generic_scalar_near_region.dat};

                \draw[thick] (axis cs:0,0,0) -- (axis cs:10,0,0);
            \end{axis}
            &
            \begin{axis}[
                width=\sizeFactor\textwidth,
                height=\sizeFactor\textwidth,
                colormap/jet,
                colorbar horizontal,
                colorbar style = {xlabel = {$\real{G_\mathrm{refl}}$}},
                xlabel = {$\rho / \lambda_0$},
                ylabel = {$z / \lambda_0$},
                view = {0}{90},
            ]
            \addplot3 [
                surf,
                point meta max=0.00065, point meta min=-0.00066,
                mesh/ordering=y varies,
                ] table[
                    x = x_by_lambda_0, y = z_by_lambda_0, z = g_tm_refl_re,
                ]
                {thesis_generic_scalar_near_region.dat};

                \draw[thick] (axis cs:0,0,0) -- (axis cs:10,0,0);
            \end{axis}
            \\
            };
        \end{tikzpicture}
        \caption{\ac{TM} case}
        \label{fig:generic_scalar_near_region_tm}
    \end{subfigure}

    \begin{subfigure}[t]{\textwidth}
        \begin{tikzpicture}
            \pgfplotsset{small}
            \matrix {
            \begin{axis}[
                width=\sizeFactor\textwidth,
                height=\sizeFactor\textwidth,
                colormap/jet,
                colorbar horizontal,
                unbounded coords = jump,
                colorbar style = {xlabel = {$20\log_{10}\abs{G_\mathrm{tot}}$}},
                xlabel = {$\rho / \lambda_0$},
                ylabel = {$z / \lambda_0$},
                view = {0}{90},
            ]
            \addplot3 [
                surf,
                point meta max=-26, point meta min=-97,
                mesh/ordering=y varies,
                ] table[
                    x = x_by_lambda_0, y = z_by_lambda_0, z = g_te_full_log,
                ]
                {thesis_generic_scalar_near_region.dat};

                \draw[thick] (axis cs:0,0,0) -- (axis cs:10,0,0);
            \end{axis}
            &
            \begin{axis}[
                width=\sizeFactor\textwidth,
                height=\sizeFactor\textwidth,
                colormap/jet,
                colorbar horizontal,
                colorbar style = {xlabel = {$\real{G_\mathrm{refl}}$}},
                xlabel = {$\rho / \lambda_0$},
                ylabel = {$z / \lambda_0$},
                view = {0}{90},
            ]
            \addplot3 [
                surf,
                point meta max=0.00065, point meta min=-0.00066,
                mesh/ordering=y varies,
                ] table[
                    x = x_by_lambda_0, y = z_by_lambda_0, z = g_te_refl_re,
                ]
                {thesis_generic_scalar_near_region.dat};

                \draw[thick] (axis cs:0,0,0) -- (axis cs:10,0,0);
            \end{axis}
            \\
            };
        \end{tikzpicture}
        \caption{\ac{TE} case}
        \label{fig:generic_scalar_near_region_te}
    \end{subfigure}

    \caption[Generic scalar Green's function for a source over dry ground]
    {Generic scalar Green's functions for a source at
    $\rsrc = \lambda_0 \uv{z}$ over dry ground.
    The left plots show the magnitude of the total field while on the right
    the real part of the extracted reflected contribution is shown.}
    \label{fig:generic_scalar_near_region}
\end{figure}




\begin{figure}
    \newcommand{\sizeFactor}{0.33}
    \newcommand{\pointMetaMax}{0.310}
    \newcommand{\pointMetaMin}{-0.811}

    \centering
    \begin{tikzpicture}
        \pgfplotsset{small}
        \matrix (m) {
            \begin{axis}[
                width=\sizeFactor\textwidth,
                height=\sizeFactor\textwidth,
                colormap/jet,
                xlabel = {$x / \lambda_0$},
                ylabel = {$z / \lambda_0$},
                view = {0}{90},
                title = $\real{G^\mathrm{EJ}_{xx}}$,
            ]
            \addplot3 [
                surf,
                mesh/ordering=y varies,
                point meta max = \pointMetaMax, point meta min = \pointMetaMin,
                ] table[
                    x = x_by_lambda_0, y = z_by_lambda_0, z = G_EJ_xx,
                ]
                {thesis_dgf_hs.dat};
                \draw[thick] (axis cs:-5,0,0) -- (axis cs:5,0,0);
            \end{axis}
            &
            \begin{axis}[
                width=\sizeFactor\textwidth,
                height=\sizeFactor\textwidth,
                colormap/jet,
                xlabel = {$x / \lambda_0$},
                ylabel = {$z / \lambda_0$},
                view = {0}{90},
                title = $\real{G^\mathrm{EJ}_{xy}}$,
            ]
            \addplot3 [
                surf,
                mesh/ordering=y varies,
                point meta max = \pointMetaMax, point meta min = \pointMetaMin,
                ] table[
                    x = x_by_lambda_0, y = z_by_lambda_0, z = G_EJ_xy,
                ]
                {thesis_dgf_hs.dat};
                \draw[thick] (axis cs:-5,0,0) -- (axis cs:5,0,0);
            \end{axis}
            &
            \begin{axis}[
                width=\sizeFactor\textwidth,
                height=\sizeFactor\textwidth,
                colormap/jet,
                xlabel = {$x / \lambda_0$},
                ylabel = {$z / \lambda_0$},
                view = {0}{90},
                title = $\real{G^\mathrm{EJ}_{xz}}$,
            ]
            \addplot3 [
                surf,
                mesh/ordering=y varies,
                point meta max = \pointMetaMax, point meta min = \pointMetaMin,
                ] table[
                    x = x_by_lambda_0, y = z_by_lambda_0, z = G_EJ_xz,
                ]
                {thesis_dgf_hs.dat};
                \draw[thick] (axis cs:-5,0,0) -- (axis cs:5,0,0);
            \end{axis}
            \\
            \begin{axis}[
                width=\sizeFactor\textwidth,
                height=\sizeFactor\textwidth,
                colormap/jet,
                xlabel = {$x / \lambda_0$},
                ylabel = {$z / \lambda_0$},
                view = {0}{90},
                title = $\real{G^\mathrm{EJ}_{yx}}$,
            ]
            \addplot3 [
                surf,
                mesh/ordering=y varies,
                point meta max = \pointMetaMax, point meta min = \pointMetaMin,
                ] table[
                    x = x_by_lambda_0, y = z_by_lambda_0, z = G_EJ_yx,
                ]
                {thesis_dgf_hs.dat};
                \draw[thick] (axis cs:-5,0,0) -- (axis cs:5,0,0);
            \end{axis}
            &
            \begin{axis}[
                width=\sizeFactor\textwidth,
                height=\sizeFactor\textwidth,
                colormap/jet,
                xlabel = {$x / \lambda_0$},
                ylabel = {$z / \lambda_0$},
                view = {0}{90},
                title = $\real{G^\mathrm{EJ}_{yy}}$,
            ]
            \addplot3 [
                surf,
                mesh/ordering=y varies,
                point meta max = \pointMetaMax, point meta min = \pointMetaMin,
                ] table[
                    x = x_by_lambda_0, y = z_by_lambda_0, z = G_EJ_yy,
                ]
                {thesis_dgf_hs.dat};
                \draw[thick] (axis cs:-5,0,0) -- (axis cs:5,0,0);
            \end{axis}
            &
            \begin{axis}[
                width=\sizeFactor\textwidth,
                height=\sizeFactor\textwidth,
                colormap/jet,
                xlabel = {$x / \lambda_0$},
                ylabel = {$z / \lambda_0$},
                view = {0}{90},
                title = $\real{G^\mathrm{EJ}_{yz}}$,
            ]
            \addplot3 [
                surf,
                mesh/ordering=y varies,
                point meta max = \pointMetaMax, point meta min = \pointMetaMin,
                ] table[
                    x = x_by_lambda_0, y = z_by_lambda_0, z = G_EJ_yz,
                ]
                {thesis_dgf_hs.dat};
                \draw[thick] (axis cs:-5,0,0) -- (axis cs:5,0,0);
            \end{axis}
            \\
            \begin{axis}[
                width=\sizeFactor\textwidth,
                height=\sizeFactor\textwidth,
                colormap/jet,
                xlabel = {$x / \lambda_0$},
                ylabel = {$z / \lambda_0$},
                view = {0}{90},
                title = $\real{G^\mathrm{EJ}_{zx}}$,
            ]
            \addplot3 [
                surf,
                mesh/ordering=y varies,
                point meta max = \pointMetaMax, point meta min = \pointMetaMin,
                ] table[
                    x = x_by_lambda_0, y = z_by_lambda_0, z = G_EJ_zx,
                ]
                {thesis_dgf_hs.dat};
                \draw[thick] (axis cs:-5,0,0) -- (axis cs:5,0,0);
            \end{axis}
            &
            \begin{axis}[
                width=\sizeFactor\textwidth,
                height=\sizeFactor\textwidth,
                colormap/jet,
                xlabel = {$x / \lambda_0$},
                ylabel = {$z / \lambda_0$},
                view = {0}{90},
                title = $\real{G^\mathrm{EJ}_{zy}}$,
            ]
            \addplot3 [
                surf,
                mesh/ordering=y varies,
                point meta max = \pointMetaMax, point meta min = \pointMetaMin,
                ] table[
                    x = x_by_lambda_0, y = z_by_lambda_0, z = G_EJ_zy,
                ]
                {thesis_dgf_hs.dat};
                \draw[thick] (axis cs:-5,0,0) -- (axis cs:5,0,0);
            \end{axis}
            &
            \begin{axis}[
                width=\sizeFactor\textwidth,
                height=\sizeFactor\textwidth,
                colormap/jet,
                xlabel = {$x / \lambda_0$},
                ylabel = {$z / \lambda_0$},
                view = {0}{90},
                title = $\real{G^\mathrm{EJ}_{zz}}$,
            ]
            \addplot3 [
                surf,
                mesh/ordering=y varies,
                point meta max = \pointMetaMax, point meta min = \pointMetaMin,
                ] table[
                    x = x_by_lambda_0, y = z_by_lambda_0, z = G_EJ_zz,
                ]
                {thesis_dgf_hs.dat};
                \draw[thick] (axis cs:-5,0,0) -- (axis cs:5,0,0);
            \end{axis}
            \\
            };
            \begin{axis}[
                anchor = north west,
                at = {(m.south west)},
                hide axis,
                scale only axis,
                height=0pt,
                width=0pt,
                colormap/jet,
                colorbar horizontal,
                point meta max=\pointMetaMax,
                point meta min=\pointMetaMin,
                colorbar style={
                    width=\textwidth,
                    xtick={-0.8,-0.7,-0.6,-0.5,-0.4,-0.3,-0.2,-0.1,0,0.1,0.2,0.3},
                    xlabel = {Real part of components in $\si{\volt\per\metre}$},
                    }
                ]
                \addplot [draw=none] coordinates {(0,0)};
            \end{axis}
        \end{tikzpicture} 
        \caption[Dyadic Green's function $\dyad{G}^\mathrm{EJ}$ for a source at
        $\rsrc = \lambda_0 \uv{z}$ over dry ground]
        {Dyadic Green's function $\dyad{G}^\mathrm{EJ}$ in the
        $y = 0$ cut plane for a source at $\rsrc = \lambda_0 \uv{z}$ over dry
        ground.}
\end{figure}


\section{Discrete Complex Image Method}


\begin{figure}
    \newcommand{\heightFactor}{0.3}
    \centering
    \begin{tikzpicture}
        \begin{axis}[
            name=plot1,
            width = \textwidth,
            height = \heightFactor\textwidth,
            xlabel = {$t$},
            xtick = {0,0.1,0.2},
            ylabel = {Random signal $y\left(t\right)$},
            grid = major,
            enlargelimits = false,
            legend entries={$\real{y}$,$\real{y}$},
            ]
            \addplot [
                blue,
                ] table [
                    x = t,
                    y = y_true_re,
                    ] {thesis_gpof_synth_signal.dat};
            \addplot [
                red,
                ] table [
                    x = t,
                    y = y_true_im,
                    ] {thesis_gpof_synth_signal.dat};
            \addplot [draw=magenta, fill=magenta, fill opacity = 0.1] coordinates {(0.1,40) (0.2,40) (0.2,-40) (0.1,-40)};
        \end{axis}
        \begin{axis}[
            name=plot2,
            at={($(plot1.south)+(0,-0.1\textwidth)$)},
            width = \textwidth,
            height = \heightFactor\textwidth,
            anchor=north,
            xlabel = {$t$},
            xtick = {0,0.1,0.2},
            ylabel = {Relative error in $\si{\decibel}$},
            grid = both,
            enlargelimits = false,
            ]
            \addplot [
                blue,
                ] table [
                    x = t,
                    y = rel_err_db,
                    ] {thesis_gpof_synth_signal.dat};
            \addplot [draw=magenta, fill=magenta, fill opacity = 0.1] coordinates {(0.1,-70) (0.2,-70) (0.2,-300) (0.1,-300)};
        \end{axis}
    \end{tikzpicture}
    \caption[]
    {Reconstruction of a random exponential series signal with known model
    order $M = \num{10}$ from parameters obtained by the \ac{GPOF} method.
    The signal is sampled with $f_\mathrm{s} = 10 f_\mathrm{max}$ at
    $N = \num{100}$ points where only the signal for $0 \leq t \leq \num{0.1}$
    is fed into the algorithm.
    The extrapolated region is visualized by colored background.}
    \label{fig:thesis_gpof_synth_signal}
\end{figure}



\begin{figure}
    \centering
    \begin{subfigure}[t]{\textwidth}
        \centering
        \newcommand{\sizeFactor}{0.47}
        \begin{tikzpicture}
            \pgfplotsset{small}
            \matrix {
                \begin{axis}[
                    width=\sizeFactor\textwidth,
                    height=\sizeFactor\textwidth,
                    colormap/jet,
                    colorbar horizontal,
                    colorbar style = {xlabel = Relative error in $\si{\decibel}$},
                    unbounded coords = jump,
                    xlabel = {$\rho / \lambda_0$},
                    ylabel = {$z / \lambda_0$},
                    view = {0}{90},
                    title = {TM case},
                ]
                \addplot3 [
                    surf,
                    mesh/ordering=y varies,
                    ] table[
                        x = rho_by_lambda_0, y = z_by_lambda_0, z = rel_err_db_tm,
                        ]
                        {thesis_dcim_near_range_error.dat};
                \end{axis}
                &
                \begin{axis}[
                    width=\sizeFactor\textwidth,
                    height=\sizeFactor\textwidth,
                    colormap/jet,
                    colorbar horizontal,
                    colorbar style = {xlabel = Relative error in $\si{\decibel}$},
                    unbounded coords = jump,
                    xlabel = {$\rho / \lambda_0$},
                    ylabel = {$z / \lambda_0$},
                    view = {0}{90},
                    title = {TE case},
                ]
                \addplot3 [
                    surf,
                    mesh/ordering=y varies,
                    ] table[
                        x = rho_by_lambda_0, y = z_by_lambda_0, z = rel_err_db_te,
                        ]
                        {thesis_dcim_near_range_error.dat};
                \end{axis}
                \\
            };
        \end{tikzpicture}
        \caption{Accuracy.}
    \end{subfigure}

    \begin{subfigure}[]{\textwidth}
        \centering
        \begin{tikzpicture}
            \pgfplotsset{small}
            \begin{axis}[
                width = \textwidth,
                height = 0.3\textwidth,
                xlabel = {$z / \lambda_0$},
                ylabel = {Number of images},
                grid = major,
                enlargelimits = false,
                legend entries = {TM case,TE case},
            ]
            \addplot[
                only marks,
                mark = o,
                color = blue,
                ]
                table [
                    x = z_by_lambda_0,
                    y = N_images_tm,
                ]
                {thesis_dcim_near_range_images.dat};
            \addplot[
                only marks,
                mark = asterisk,
                color = red,
            ]
            table [
                x = z_by_lambda_0,
                y = N_images_te,
                ]
                {thesis_dcim_near_range_images.dat};
            \end{axis}
        \end{tikzpicture}
        \caption{Number of complex images.}
    \end{subfigure}

    \caption{Generic scalar Green's function in air over dry ground via
    \ac{DCIM} for $\rsrc = \lambda_0 \uv{z}$.}
    \label{fig:dcim_near_range}
\end{figure}



\section{Fast Multipole Method}

\begin{figure}
    \centering
    \begin{tikzpicture}
        \pgfplotsset{small}
        \begin{axis}[
            width=0.6\textwidth,
            colormap/jet,
            colorbar,
            colorbar style = {ylabel = {Relative error in $\si{\decibel}$}},
            xlabel = {$x / \lambda_0$},
            ylabel = {$y / \lambda_0$},
            xmode = log,
            ymode = log,
            view = {0}{90},
        ]
        \addplot3 [
            surf,
            mesh/ordering=y varies,
            ] table[
                x = x_by_lambda_0, y = y_by_lambda_0, z = rel_err_db,
                ]
                {thesis_fmm_standard_accuracy.dat};
        \end{axis}
    \end{tikzpicture}
    \caption[Accuracy of standard \ac{FMM} implementation with real sources]
    {Accuracy of standard \ac{FMM} implementation with real sources.
    The cubical boxes are of side length $w = \lambda_0$.
    A total $\num{3600}$ observation points are distributed in the Cartesian
    $z = 0$ plane.
    Due to the logarithmic spacing each source point lies within an own
    group.
    The total $\num{500}$ source points are contained within a single group at
    $\fmmMultiIndex = \left(0,0,20\right)$.}
    \label{fig:thesis_fmm_standard_accuracy}
\end{figure}

