\chapter{Computer Implementation}

\section{Numerical Integration of Sommerfeld Integrals}

\section{Discrete Complex Image Method}

\subsection{\ac{GPOF}}

\cite{Hua1989}
\cite{Sarkar1995}
\cite{MohammadiGhazi2016}
\cite{BergFriedlander:2008, spgl1site}

\section{Fast Multipole Method}

Within this thesis the goal is not to develop a fully-featured \ac{FMM}
implementation, \eg, for the use in an electromagnetic solver.
It is rather desired to have a simple but flexible test code to examine
different possibilities to adapt the \ac{FMM} to the half-space Green's
function.
With this key requirement in mind an experimental version of the \ac{FMM}
is implemented which is restricted in the following aspects.
\begin{itemize}
	\item Only the scalar case is considered.
	\item Only $\diracDelta$-functions are used as basis and testing functions.
	\item The locations of basis and testing functions are set arbitrarily or 
	according to a specific test case. No real body-like structures are
	considered.
	\item Near interactions are not considered as their conventional \ac{MoM}
	treatment is straightforward.
\end{itemize}

Even though the described setup is conceptually similar to the examples
provided in a paper by \textcite{Hansen2013}, it is found that the subject
of this thesis requires a more versatile approach.
Therefore, the above specifications are merged into the prescription of the
algorithm given by the popular paper by \textcite{Coifman1993}.

Throughout this thesis the regular grouping scheme elucidated by
\cref{fig:fmm_geometry} is assumed for all considerations regarding the \ac{FMM}.
All groups are cubical with side length $w$ and uniquely identified by a
multiindex $\fmmMultiIndex \in \Z^3$ defined as
\begin{equation}
	\fmmMultiIndex \left(\pvec{r}\right) \coloneqq \lfloor \pvec{r} / w \rfloor
\end{equation}
where the floor function is defined to operate component-wise on the vector
$\pvec{r}$ given in Cartesian coordinates.

The centers of certain source and observation groups are denoted by 
$\fmmGroupCenter{\fmmMultiIndex^\prime}$ and 
$\fmmGroupCenter{\fmmMultiIndex}$, respectively.
The connection between two group centers is denoted by
$\fmmDiffGroupCenters{\fmmMultiIndex}{\fmmMultiIndex^\prime} \coloneqq \fmmGroupCenter{\fmmMultiIndex} - \fmmGroupCenter{\fmmMultiIndex^\prime}$.

Within each group the source or observation points belonging to this group
are consecutively numbered by an index $\alpha \in \N$.
Globally, all points are identified by a unique index $n \in \N$.
The small example given in \cref{tab:fmm_grouping} illustrates
the chosen indexing scheme.

The implemented version of the \ac{FMM} does impose no restrictions to the
locations of basis and testing functions other than the fulfillment of the
separation criterion which makes the \ac{FMM} applicable.
The basis and testing functions are especially not assumed to be arranged
according to Galerkin's method \cite[p.~7]{Harrington1993}, where the
self-coupling case regularly occurs.

Due to this freedom it turns out to be advantageous to establish two separate
correspondences as the one given by \cref{tab:fmm_grouping}: one for source
points and one for observation points.

\begin{table}[hbt]
	\centering
	\begin{tabular}{ccc}
		\toprule%
		$\fmmMultiIndex$ & $\alpha$ & $n$ \\
		\midrule
		$\left(0, 0, 0\right)$ & $1$ & $1$ \\
		                       & $2$ & $2$ \\
		                       & $3$ & $3$ \\
		\midrule
		$\left(5, 0, 2\right)$ & $1$ & $4$ \\
		                       & $2$ & $5$ \\
		                       & $3$ & $6$ \\
		                       & $4$ & $7$ \\
		\midrule
		$\left(0, 3, 0\right)$ & $1$ & $8$ \\
		                       & $2$ & $9$ \\
		                       & $3$ & $10$ \\
		\bottomrule
	\end{tabular}

	\caption[\acs{FMM} grouping scheme]
	{\acs{FMM} grouping scheme. The setup in this example has three groups
	containing three, four, and another three points, respectively.}

	\label{tab:fmm_grouping}

\end{table}
