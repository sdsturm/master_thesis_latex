\chapter{Introduction}
\label{ch:introduction}

\section{Background}

The references at the end of this document thematically are all part of an area
which could be enveloped as \emph{electromagnetic theory and practical field
computation.}
The overall contribution of some of these publications to the area might be
marginal.
Some might be beneficial in solving certain problems at hand.
Some might be seminal \cite{Sommerfeld1909}.
One---without any doubt---\emph{is} epochal
\cite{maxwell1865}.

Although it is very unlikely these days that anybody will actually consider the
original set of twenty equations published by the great man himself in 1865
\cite{maxwell1865} when using the term \enquote{Maxwell's equations} ---
it was to the Scottish mathematician and physicist to first bring together the
pieces of insight gained by other major names during the early 19th century and
to supplement the fragments to a theory of utmost elegance and predictive power.

In defiance of the elegance and relative simplicity of electromagnetic theory
the field problem can only be solved analytically for a finite set of the most
simple and highly symmetrical problems.
Even some of those problems have forced past generations to apply and in some
cases even develop sophisticated mathematical formalisms \cite{Felsen1994}.
In modern times, however, the advances in powerful computers enables the use of  
numerical methods to solve the field problem for more real-world scenarios in
principle up to the accuracy of machine precision.
Some of these methods were even known for a long time, but had to await
practical applicability until major computational power was available.

A particularly pleasant circumstance, however, is that many of the ancestral
analytical efforts have by no means become obsolete --- they rather can be used
beneficially in powerful numerical methods.
The example which is subject to this thesis is the use of fundamental solutions
\textemdash so-called Green's functions \textemdash within the framework
provided by the \ac{MoM} \cite{Harrington1993}.

The particular problem motivating this thesis is to find a combination of
different numerical methods which all employ analytical results in a very
distinguished way. Namely, the Green's function of the so-called half-space
problem shall be used in conjunction with the \ac{FMM} in order to accelerate
the numerical solution of field problems formulated in terms of integral
equations.

\section{Sommerfeld Half-Space Problem and Multilayered Media}

An outstanding both for the prosperous use of analytical techniques and for the
gainful use of so obtained results in numerical techniques is the Green's 
function of the half-space problem and closely related multilayered media.

The first goes back to the seminal publications of Arnold Sommerfeld in the
early days of radio and antennas \cite{Sommerfeld1909}.
He first considered a vertically oriented \ac{HD}, residing at the interface
formed by semi-unbounded free-space and the surface of the earth.
The latter is assumed to be flat and to possess lossy dielectric material
properties.
With his work Sommerfeld mathematically verified the assumed existence of
a special kind of surface wave \cite{Zenneck1907} and examined how the separate
constituents of the total wave field correspond to purely mathematical
attributes of the underlying formalism.
This, in conjunction to the first-place development of this very formalism is
the cause why Sommerfeld's 1909 paper is rewarded as pioneering down to the
present day \cite{Michalski2015, Michalski2016b, Mosig2021}.
Sommerfeld later elaborated more on the topic
\cite{Sommerfeld1926,Sommerfeld1964}, \eg, generalizing it to the \ac{HED}
case and to a situation where the dipole is located above the interface
At least in the 1926 paper \cite{Sommerfeld1926} a small but controversially
discussed sign error from the original paper \cite{Wait1998} was fixed.

Sommerfeld's formalism was later generalized to arbitrary solution domains
compound by several plane layers \cite{Felsen1994, Michalski1990},
possibly even ones with anisotropic material parameters and \ac{PEC}, \ac{PMC}
as well as impedance boundary conditions \cite{Hsu1993, Michalski2005}.

\fxnote{Sommerfeld originally \cite{Sommerfeld1909} solved for the vector
potential of the \ac{VED} (\enquote{Hertz potential})}

\section{Numerical Evaluation of Sommerfeld Integrals}

General (no PDF):
\cite{mosig1982}

Path deformation
\cite{Michalski1985a}

Tails, \ac{PE}
\cite{Michalski1998}
\cite{Mosig2012}
\cite{mosig2013}
\cite{Michalski2016a}
\cite{Golubovic2012}

Head DE rules
\cite{golubovicniciforovic2011}

Head detour semi-elliptical
\cite{GayBalmaz1997}

Head detour roof-top
\cite{Michalski2015a}

MPM
\cite{MengtaoYuan2006a}

\ac{SIP}, \ac{EIP} only for $\rho$ up to a few hundred wavelenghts \cite[14]{Michalski2016b}
Reason: not the tail \ac{PE} but rapid oscillations in head

\section{Closed Form Approximations}

\section{Integral Equation Techniques and the Method of Moments}

\cite{Harrington1993}

Integral equations arising from boundary value problems in \ac{CE} can be
discretized by subdomain basis functions, \eg, the popular \ac{RWG} functions
\cite{Rao1982}.

Direct solver: $N^3$
Iterative solver: $N^2$

\section{Fast Multipole Method}

\subsection{Motivation}

Due to the global nature of the involved Green's functions the discretization
of \ac{MoM}-based discretization of integral equations lead to a dense \ac{LSE}.
These systems are most commonly solved using iterative solvers in which the
system matrix is applied to a several, recursively generated, excitation
vectors until the solution has converged to the desired accuracy.
Popular examples for iterative solvers are multiple versions of the \ac{CG}
or the \ac{GMRES} \cite{Strang2010}.

If now, in a Galerkin-type \ac{MoM} scheme, $N$ denotes the number of expansion
and testing functions, each single matrix-vector multiplication exhibits a
complexity of $\landau{N^2}$ in terms of operations as well as in terms of
memory. Since this becomes prohibitive for large $N$ as occurring in
field computations of many real-world problems, an extensive set of fast
algorithms has been developed which all aim to accelerate matrix-vector
multiplication \cite[chapter 11]{Jin2015}.

One such fast method is the \ac{FMM} and, as an extension of the latter, the
\ac{MLFMA}. In principle these algorithms are capable to reduce the complexity
to $\landau{N^{3/2}}$ and even $\landau{N \log N}$, respectively.
As opposed to some other fast algorithms, the \ac{FMM} is problem
specific. 
Nevertheless, 2D \cite{Rokhlin1990} and 3D \cite{Rokhlin1993, Coifman1993}
versions of the \ac{FMM} have been developed for the major equation in
full-wave \ac{CE}, the Helmholtz equation.
Today the use of the \ac{MLFMA} is state of the art in the treatment of
large-scale problems with integral equation techniques\footnote{Interestingly,
in the early days of the \ac{FMM} even its creators assumed that it would not be
beneficial to improve the method to a $\landau{N \log N}$ algorithm by
applying the scheme recursively several times. The reason is they assumed
\enquote{problems of practicable size} to be not larger than
$N \le \num{1000000}$ \cite{Rokhlin1993}.}
The involved algorithms, data structures and the choice of important
parameters have been extensively studied and  are well-documented
\cite{Chew2001}.

\subsection{Principle}

The starting point in the development of the \ac{FMM} is the Green's function
of free space.
The core idea, interpreted in terms of electrodynamics, is as follows.
In order to solve the field problem for a general distribution of $N$ sources
of radiation --- we may assume $N$ \acp{HD} radiating into free space ---
one has to evaluate the corresponding Green's function $N^2$ times.
This is due to the interaction of every \ac{HD} with every other one.
If now the \acp{HD} are not distributed equally in space but somewhat are
\emph{clustered} the idea of a multipole expansion 
\cite[9.7]{Jackson2013} \cite[7.10]{vanBladel2007} comes into place.
The claim is that in a cluster of \acp{HD} each individual \ac{HD} will not
\enquote{see} every single \ac{HD} in a cluster of \acp{HD} which is far away.
Instead, it will be sufficient to consider only bigger entities, namely the
whole clusters, in terms of far interaction while the near interactions
between individual \acp{HD} need to be considered only within the clusters.
Concisely formulated the \ac{FMM} consists of the following three steps. 
\begin{enumerate}
	\item The effects of all individual \acp{HD} within a cluster of sources
	are \emph{aggregated} to a common center within the respective group. 
	\item The aggregated effect is \emph{translated} to the common reference
	point of a far-away group of, likewise clustered, observing \acp{HD}.
	\item Within the group of observing \acp{HD} the translated quantity is
	redistributed or \emph{disaggregated} to all individual receiving \acp{HD}.
\end{enumerate}

These steps somewhat correspond to a factorization of the Green's function
into a product of three factors.

The same principle sketched here for point-like sources can easily be extended
to arbitrary basis and testing functions within the \ac{MoM} by invoking
the linearity of Maxwell's equations and the superposition principle.

Since the described approach only works for far-interactions of properly
clustered groups of \acp{HD} or --- more general --- subdomain basis functions 
of local support, all the possible near-interactions need to be treated
separately.
This can be done applying the conventional \ac{MoM} approach to compute all
nearby interactions.
The effect here is, that the \ac{FMM} removes all entries in the system matrix
which would not be centered around its diagonal within the standard \ac{MoM}.
The result is a sparse and banded matrix, similar to those resulting from
techniques based on differential equations, such as the \ac{FEM} or certain
forms of \acp{FDM}.
This resulting sparse matrix is then much easier to treat than the corresponding
full \ac{MoM} matrix. At the same time the obtained algorithm benefits from the
reduced number of unknowns as compared to differential equation techniques.

\subsection{Extension to Source Points in Complex Space}

The use of complex vectors \cite[Chapter 1]{Lindell2000} is omnipresent within
the study of time-harmonic electromagnetic fields.
However, vectors in $\C^3$ are usually \emph{electromagnetic} quantities like
vector potentials and the fields themselves.
In contrast, (arbitrary) complex spatial coordinates are rather uncommon so
far\footnote{Counterexamples are found, \eg, within the context of Lorentz
transforms between moving frames of reference, where angles of rotation may be
imaginary \cite[Chapter~11]{Jackson2013}.}.
An example use case in which the concept of complex space plays a prominent role
is given in a paper by \textcite{Hansen2009}, where a new perspective on the
classical Huygens principle based on complex source locations is presented. 

It is well-known that if the Green's function of the scalar Helmholtz equation
for the free-space case describes a spherical wave for the usual case where
the distance between source and observation point is a real number.
This pattern changes if this distance is complex.
In this case the magnitude of the wave field becomes dependent on direction.
This kind of wave field is commonly called Gaussian beam or \ac{CSB}.
Albeit the concept of \emph{complex space} is somewhat abstract
\cite[Chapter~1]{Lindell2000}, an intuitive and commonly employed definition of
\emph{complex length} is, that such a quantity should have a positive real part
\cite{Hansen2009, Hansen2013}.

Concerning the combination of the \ac{FMM} with sources situated in complex
space, it appears obvious that this kind of application was not what the
creators originally had in mind.
The popular review paper by \textcite{Coifman1993}, for example, does not make
any indications towards this direction.
Interestingly enough, in the mathematically rigorous treatment of the 3D
\ac{FMM} for the Helmholtz equation \cite{Rokhlin1993}, the essential
Theorem~2.2 is formulated in terms of complex spatial coordinates and even
complex angles.
However, these assumptions are dropped as early in the following as in
Theorem~2.3, whose requirements assume real spatial quantities and leave only
the wavenumber in the complex domain.
The same assumptions underlie the mathematically oriented publication Concerning
the 2D \ac{FMM} \cite{Rokhlin1990}.

From these observations it appears obvious, that the \ac{FMM} development
made no efforts towards an applicability to sources (or observation points)
situated in whatever form of complex space.
Actually in most cases of application even the wavenumber is simply the one of
free space and is therefore real quantity.

Nevertheless, a combination of the concept of \acp{CSB} with the \ac{FMM} can
provide benefits.
Employing some earlier results \cite{Hansen2009}, it was shown by
\textcite{Hansen2013}, that the introduction of a well-defined
\emph{imaginary shift} applied to the basis functions in a \ac{FMM} setup
results in a modified \ac{FMM} translation operator which shows a significantly
directive pattern.
This so-called Gaussian translation operator allows to neglect some plane wave
directions involved in the expansion throughout the translation step
and therefore renders the resulting algorithm computationally more effective
than a \ac{FMM} using the standard translation operator.

It should be noted that the requirements in the derivation of the Gaussian
translation operator and also the corresponding proof 
\cite[Appendix]{Hansen2013} do not assume general source points located in
complex space defined in a sense that all three spatial coordinates might
be arbitrary complex numbers.
The imaginary part of the source position vector is instead selected by a given
intention regarding the sharpness of the resulting \ac{CSB}.
Consequently, the parameter controlling the imaginary shift of source points
is labeled \emph{beam parameter} by the author.

\fxnote{\cite{Borries2014}}
\fxnote{\cite{eibert2021}}

In the course of the literature research for this thesis no thoroughly and 
mathematically rigorous treatment was found concerning the combination
of the \ac{FMM} principle with actual complex source points located at
\emph{arbitrary} \enquote{positions} in a complex version of Euclidean space.

However, there are records of some attempts to do so using a standard \ac{FMM}
implementation and without the aim of extending the underlying theorems and
proofs \cite{Rokhlin1993}.
A rather heuristic discussion of this topic is given in a paper by
\textcite{Geng2001}.
It is empirically shown that the use of complex source points drastically
deteriorates the convergence of the central addition theorem on which the
\ac{FMM} is based.
The number of terms in the corresponding series expansion is closely
related to the number of quadrature points required to numerically integrate
the plane-wave expansion --- the second main identity within the
\ac{FMM}.
For this reason the authors did not pursue the approach of using complex images
further and instead directed their attention to approximate forms of the
Green's function using a single real image source for all far interaction
computed using the \ac{FMM} \cite{Geng1999, Geng2000}.

\subsection{Adaption to Half-Space Green's Functions}

\cite{Geng2000}
\cite{Liu2002}
\cite{Eibert2018}
\cite{Eibert2018a}
\cite{Eibert2019}
\cite[Chapter~8.13.2]{parini2020}
