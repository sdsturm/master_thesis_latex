\chapter*{Abstract}
\addcontentsline{toc}{chapter}{Abstract}

Arnold Sommerfeld's integral representation of the Green's functions of
half-space and planar multilayered solution environments still---although more 
than a century of age---is the prevalent formalism in this field of research.
As no closed form solutions of these Sommerfeld integrals exist for any
problem of practical interest, an extensive set of analytical and numerical
methods is the product of long-term research.
Using those tools, a trade-off has to be made between reliable accuracy and
computational efficiency.
A highly effective yet not absolutely reliable method is the \ac{DCIM}
which utilizes the Sommerfeld identity together with a series of complex spatial
distances in order to approximate Sommerfeld integrals in closed form.
The terms of this series are interpreted as complex images.

The so-obtained Green's functions can be used in integral equation
techniques within the framework of the \ac{MoM}.
As \ac{MoM} solutions may be of prohibitive computational cost for electrically
large problems, the \ac{FMM} is a well-established tool to reduce the cost
in \ac{MoM}-based formulations which use the Green's function of free space.

This thesis presents a unified and comprehensive treatment of the preliminaries
that are expected to be needed in further research towards the development
of improved or extended \ac{FMM}-like algorithms for acceleration of 
\ac{MoM}-based formulation which instead contain the Green's function of
half-space or eventually even the multilayered solution environments.
The central mathematical tools are examined with regard to their compatibility
with image sources situated in complex space.
All required components are implemented and thoroughly numerically verified
against known analytical solutions or using the fundamental electromagnetic
principles of reciprocity and duality.
The core of this software library is formed by a set of routines for the
computation of highly accurate reference solutions by
advanced extrapolation-accelerated numerical integration techniques.
For illustration, a comprehensive set of example numerical results is presented.

\acresetall
