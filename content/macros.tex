%%%%%%%%%%%%%%%%%%%%%%%%%%%%%%%%%%%%%%%%%%%%%%%%%%%%%%%%%%%%%%%%%%%%%%%%%%%%%%%%
% Abbreviations
%%%%%%%%%%%%%%%%%%%%%%%%%%%%%%%%%%%%%%%%%%%%%%%%%%%%%%%%%%%%%%%%%%%%%%%%%%%%%%%%
\newcommand{\ie}{i.\,e.\xspace}%
\newcommand{\eg}{e.\,g.\xspace}%
\newcommand{\wrt}{w.\,r.\,t.\xspace}%

%%%%%%%%%%%%%%%%%%%%%%%%%%%%%%%%%%%%%%%%%%%%%%%%%%%%%%%%%%%%%%%%%%%%%%%%%%%%%%%%
% Acronyms
%%%%%%%%%%%%%%%%%%%%%%%%%%%%%%%%%%%%%%%%%%%%%%%%%%%%%%%%%%%%%%%%%%%%%%%%%%%%%%%%

\DeclareAcronym{CE}{
    short = CE,
    long = computational electromagnetics
    }
\DeclareAcronym{ODE}{
    short = ODE,
    long = ordinary differential equation,
    short-plural = s,
    long-plural = s
    }
\DeclareAcronym{LSE}{
    short = LSE,
    long = linear system of equations,
    }
\DeclareAcronym{PDE}{
    short = PDE,
    long = partial differential equation
    }
\DeclareAcronym{PEC}{
    short = PEC,
    long = perfect electric conductor
    }
\DeclareAcronym{PMC}{
    short = PMC,
    long = perfect magnetic conductor
    }
\DeclareAcronym{FMM}{
    short = FMM,
    long = fast multipole method
    }
\DeclareAcronym{MLFMA}{
    short = MLFMA,
    long = multilevel fast multipole algorithm
    }
\DeclareAcronym{DCIM}{
    short = DCIM,
    long = discrete complex image method
    }
\DeclareAcronym{GPOF}{
    short = GPOF,
    long = generalized penicl-of-function method
    }
\DeclareAcronym{CSB}{
    short = CSB,
    long = complex-source beam,
    short-plural = s,
    long-plural = s
    }
\DeclareAcronym{EIT}{
    short = EIT,
    long = exact image theory
    }
\DeclareAcronym{MoM}{
    short = MoM,
    long = method of moments
    }
\DeclareAcronym{FDM}{
    short = FDM,
    long = finite difference method,
    short-plural = s,
    long-plural = s,
    }
\DeclareAcronym{FEM}{
    short = FEM,
    long = finite element method,
    }
\DeclareAcronym{HD}{
    short = HD,
    long = Hertzian dipole,
    short-plural = s,
    long-plural = s,
    }
\DeclareAcronym{VED}{
    short = VED,
    long = vertical electric dipole,
    short-plural = s,
    long-plural = s,
    }
\DeclareAcronym{HED}{
    short = HED,
    long = horizontal electric dipole,
    short-plural = s,
    long-plural = s,
    }
\DeclareAcronym{VMD}{
    short = VMD,
    long = vertical magnetic dipole,
    short-plural = s,
    long-plural = s,
    }
\DeclareAcronym{HMD}{
    short = HMD,
    long = horizontal magnetic dipole,
    short-plural = s,
    long-plural = s,
    }
\DeclareAcronym{TL}{
    short = TL,
    long = transmission line,
    short-plural = s,
    long-plural = s,
    }
\DeclareAcronym{TLGF}{
    short = TLGF,
    long = transmission line Green's function,
    short-plural = s,
    long-plural = s,
    }
\DeclareAcronym{TM}{
    short = TM,
    long = transversally magnetic
    }
\DeclareAcronym{TE}{
    short = TE,
    long = transversally electric
    }
\DeclareAcronym{CG}{
    short = CG,
    long = conjugate gradient method
    }
\DeclareAcronym{GMRES}{
    short = GMRES,
    long = generalized minimal residual algorithm
    }
\DeclareAcronym{RWG}{
    short = RWG,
    long = Rao Wilton Glisson
    }
\DeclareAcronym{SIP}{
    short = SIP,
    long = Sommerfeld integration path
    }
\DeclareAcronym{EIP}{
    short = EIP,
    long = extended integration path
    }
\DeclareAcronym{PE}{
    short = PE,
    long = partition-extrapolation
    }

%%%%%%%%%%%%%%%%%%%%%%%%%%%%%%%%%%%%%%%%%%%%%%%%%%%%%%%%%%%%%%%%%%%%%%%%%%%%%%%%
% Theorem environments
%%%%%%%%%%%%%%%%%%%%%%%%%%%%%%%%%%%%%%%%%%%%%%%%%%%%%%%%%%%%%%%%%%%%%%%%%%%%%%%%

\theoremstyle{definition}
\newtheorem{definition}{Definition}[section]

\theoremstyle{theorem}
\newtheorem{theorem}{Theorem}[section]

\theoremstyle{corollary}
\newtheorem{corollary}{Corollary}[theorem]

\theoremstyle{lemma}
\newtheorem{lemma}[theorem]{Lemma}

\theoremstyle{remark}
\newtheorem*{remark}{Remark}


%%%%%%%%%%%%%%%%%%%%%%%%%%%%%%%%%%%%%%%%%%%%%%%%%%%%%%%%%%%%%%%%%%%%%%%%%%%%%%%%
% Own mathematical commands
%%%%%%%%%%%%%%%%%%%%%%%%%%%%%%%%%%%%%%%%%%%%%%%%%%%%%%%%%%%%%%%%%%%%%%%%%%%%%%%%

% Physical and geometrical vectors
\newcommand{\pvec}[1]{\mathbfit{#1}}     	% general "physical" vector and vector in frequency domain
\newcommand{\tdvec}[1]{\bm{\mathfrak{#1}}}	% time domain
\newcommand{\uv}[1]{\hat{\pvec{#1}}} 		% unit vector
\newcommand{\dyad}[1]{\overline{\pvec{#1}}}
\renewcommand{\vec}[1]{\mathsfbfit{#1}} 	% n-dimensional vector
\newcommand{\mat}[1]{\mathsfbfit{#1}} 		% <n x n> matrix

% Operators
\newcommand{\op}[1]{\mathcal{#1}} 			% operator that takes a scalar-valued function
\newcommand{\vecop}[1]{\bm{\mathcal{#1}}} 	% operator that takes a vector-valued function
\DeclareMathOperator{\laplace}{\Delta}		% scalar Laplacian operator
\DeclareMathOperator{\laplaceTrans}{\laplace_\mathrm{t}}	% transversal Laplacian operator
\DeclareMathOperator{\gradient}{\nabla}		% gradient operator
\DeclareMathOperator{\divergence}{\nabla\cdot}		% divergence operator
\DeclareMathOperator{\curl}{\nabla\times}			% curl operator
\newcommand{\abs}[1]{\left\vert#1\right\vert}	% absolute value
\newcommand{\norm}[1]{\left\Vert #1 \right\Vert}
\newcommand{\euclideanNorm}[1]{\norm{#1}_2} % Euclidean norm
\newcommand{\complexLength}[1]{\norm{#1}_\mathrm{c}} % Euclidean norm
\newcommand{\pd}[2][]{\frac{\partial#1}{\partial#2}} 	% partial derivative (function as an optional argument)
\newcommand{\nd}[2][]{\frac{\dd#1}{\dd#2}} 	% normal derivative (function as an optional argument)
\newcommand{\pdsec}[1]{\frac{\partial^2}{\partial {#1}^2}}	% 2nd partial derivative
\newcommand{\pdsecmixed}[2]{\frac{\partial^2}{\partial {#1} \partial {#2}}}	% 2nd mixed partial derivative
\newcommand{\landau}[1]{\mathcal{O} \left( #1 \right)}
\newcommand{\cmplxlength}[1]{\abs{#1}_\mathrm{c}}
\newcommand{\sommerfeldIntegral}[2]{\mathcal{S}_{#1} \left\{#2\right\}}

% Misc
\newcommand{\e}{\mathrm{e}}					% Euler's number
\renewcommand{\exp}[1]{\e^{#1}\,}			% exponential function
\renewcommand{\pi}{\uppi}
\newcommand{\dd}{\mathrm{d}}  				% the differential symbol is actually an operator
\DeclareMathOperator{\const}{const.}

% Complex numbers
\newcommand{\im}{\mathrm{j}}
%\renewcommand{\Re}{\mathrm{Re}\;}
%\renewcommand{\Im}{\mathrm{Im}\;}
\newcommand{\real}[1]{\Re\mathrm{e}\left(#1\right)}
\newcommand{\imag}[1]{\Im\mathrm{m}\left(#1\right)}

% Blackboard font: for special sets of numbers
\newcommand{\C}{\mathbb{C}}	% complex numbers
\newcommand{\R}{\mathbb{R}}	% real numbers
\newcommand{\N}{\mathbb{N}}	% natural numbers
\newcommand{\Z}{\mathbb{Z}}	% integers
\newcommand{\F}{\mathbb{F}}	% a general field

% Special functions
\DeclareMathOperator{\diracDelta}{\updelta}
\newcommand{\besselj}[1]{\operatorname{J}_{#1}}
\newcommand{\sphbesselj}[1]{\operatorname{j}_{#1}}
\newcommand{\besselh}[2]{\operatorname{H}_{#2}^{\left( #1 \right)}}
\newcommand{\sphbesselh}[2]{\operatorname{h}_{#2}^{\left( #1 \right)}}
\newcommand{\legendrepl}[1]{\operatorname{P}_{#1}}

% Custom commands for this thesis
\newcommand{\robs}{\pvec{r}}
\newcommand{\rsrc}{\pvec{r}^\prime}
\newcommand{\rdiff}{\robs-\rsrc}
\newcommand{\rdiffabs}{\abs{\rdiff}}
\newcommand{\ewaldintegral}{\oiint\limits_{S^2} \dd^2 \uv{k} \, }
\newcommand{\te}{\mathrm{TE}}
\newcommand{\tm}{\mathrm{TM}}
\newcommand{\tx}{\mathrm{TX}}
\newcommand{\isrc}{\mathrm{i}}
\newcommand{\vsrc}{\mathrm{v}}

\newcommand{\epsr}[1][]{\varepsilon_{\mathrm{r}#1}} 	% relative permittivity (layer index as an optional argument)
\newcommand{\mur}[1][]{\mu_{\mathrm{r}#1}} 	% relative permeability (layer index as an optional argument)

% FMM definitions
\newcommand{\fmmDiffGroupCenters}{\pvec{t}_{m,m^\prime}}
\newcommand{\fmmGroupCenter}[1]{\pvec{c}_{#1}}
\newcommand{\fmmSrcGroupCenter}{\pvec{c}_{m^\prime}}
\newcommand{\fmmObsGroupCenter}{\pvec{c}_{m}}
\newcommand{\fmmMultiIndex}{\vec{m}}

